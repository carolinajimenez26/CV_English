%%%%%%%%%%%%%%%%%
% This is an sample CV template created using altacv.cls
% (v1.1.4, 27 July 2018) written by LianTze Lim (liantze@gmail.com). Now compiles with pdfLaTeX, XeLaTeX and LuaLaTeX.
% 
%% It may be distributed and/or modified under the
%% conditions of the LaTeX Project Public License, either version 1.3
%% of this license or (at your option) any later version.
%% The latest version of this license is in
%%    http://www.latex-project.org/lppl.txt
%% and version 1.3 or later is part of all distributions of LaTeX
%% version 2003/12/01 or later.
%%%%%%%%%%%%%%%%

%% If you need to pass whatever options to xcolor
\PassOptionsToPackage{dvipsnames}{xcolor}

%% If you are using \orcid or academicons
%% icons, make sure you have the academicons 
%% option here, and compile with XeLaTeX
%% or LuaLaTeX.
% \documentclass[10pt,a4paper,academicons]{altacv}

%% Use the "normalphoto" option if you want a normal photo instead of cropped to a circle
% \documentclass[10pt,a4paper,normalphoto]{altacv}

\documentclass[10pt,a4paper]{altacv}
%% AltaCV uses the fontawesome and academicon fonts
%% and packages. 
%% See texdoc.net/pkg/fontawecome and http://texdoc.net/pkg/academicons for full list of symbols.
%% 
%% Compile with LuaLaTeX for best results. If you
%% want to use XeLaTeX, you may need to install
%% Academicons.ttf in your operating system's font 
%% folder.


% Change the page layout if you need to
\geometry{left=1cm,right=9cm,marginparwidth=6.8cm,marginparsep=1.2cm,top=1.25cm,bottom=1.25cm,footskip=2\baselineskip}

% Change the font if you want to.

% If using pdflatex:
\usepackage[T1]{fontenc}
\usepackage[utf8]{inputenc}
\usepackage[default]{lato}

% If using xelatex or lualatex:
% \setmainfont{Lato}

% Change the colours if you want to
\definecolor{Mulberry}{HTML}{72243D}
\definecolor{SlateGrey}{HTML}{2E2E2E}
\definecolor{LightGrey}{HTML}{666666}
\definecolor{Green}{HTML}{318E57}
\definecolor{Blue}{HTML}{1C4551}
\definecolor{Purple}{HTML}{6F1F91}
\colorlet{heading}{Blue}
\colorlet{accent}{Purple}
\colorlet{emphasis}{SlateGrey}
\colorlet{body}{LightGrey}

% Change the bullets for itemize and rating marker
% for \cvskill if you want to
\renewcommand{\itemmarker}{{\small\textbullet}}
\renewcommand{\ratingmarker}{\faCircle}
%% sample.bib contains your publications
\addbibresource{sample.bib}

\usepackage[colorlinks]{hyperref}

\begin{document}

\name{CAROLINA JIMÉNEZ GÓMEZ}
\tagline{SOFTWARE ENGINEER}
% \photo{2.8cm}{CJG}
\personalinfo{%
  % Not all of these are required!
  % You can add your own with \printinfo{symbol}{detail}
  \email{carolina.jimenez.g@gmail.com }
%   \phone{000-00-0000}
%   \mailaddress{Elambilakkal(Hose),Calicut airport(Po),Kondotty, Malappuram }
%   \location{Keral ,India}
  \homepage{carolinajimenez26.github.io}
  \twitter{@carojimenez26_}
  \linkedin{linkedin.com/in/carolinajimenez26}
  \github{github.com/carolinajimenez26}
  %% You MUST add the academicons option to \documentclass, then compile with LuaLaTeX or XeLaTeX, if you want to use \orcid or other academicons commands.
%   \orcid{orcid.org/0000-0000-0000-0000}
}

%% Make the header extend all the way to the right, if you want. 
\begin{fullwidth}
\makecvheader
\end{fullwidth}

%% Depending on your tastes, you may want to make fonts of itemize environments slightly smaller
% \AtBeginEnvironment{itemize}{\small}


%% Provide the file name containing the sidebar contents as an optional parameter to \cvsection.
%% You can always just use \marginpar{...} if you do
%% not need to align the top of the contents to any
%% \cvsection title in the "main" bar.
\cvsection[page1sidebar]{Experience}

\cvevent{Software Engineer}{\href{https://www.digitalproductschool.io/index}{Digital Product School}}{September 2018 -- November 2018}{Munich, Germany}
\begin{itemize}

\item Worked in a diverse team consisting in Software Engineers, Product Managers, Interaction Designer and AI Engineer.

\item Researched Truck Driving problem in Germany, including interviews, ideation and architecture design for a prototype.

\item Developed a mobile application which allows truck drivers to share information in real time such as points of interest and text messaging. React Native and Firebase were used.
  
\end{itemize}

\divider

\cvevent{Undergraduate Researcher}{\href{http://judge.utp.edu.co/}{Sirius Research Group}}{July 2017 -- July 2018}{Pereira, Colombia}
\begin{itemize}
\item  Researched about tractographies, including comparisons between different tools to generate them and comparisons between different tractography's algorithms. 

\item Developed a test software that automates the processes for tractography generation. 

\item Created a web application using Django and Bootstrap. MRtrix tool was used with global tractography algorithm.
    
\end{itemize}

\divider

\cvevent{Research Assistant}{\href{http://judge.utp.edu.co/}{Sirius Research Group}}{February 2016 -- August 2017}{Pereira, Colombia}
\begin{itemize}

\item Researched about Digital Image Processing using Python with FSL, a library to analyse brain imaging data.

\item Documented the libraries that were used in the project and how to install them.

\item Helped with the development of a program that calculates brain densitometries.

\item Created a web application with Node.js, HTML/CSS, Materialize and JavaScript. 

\end{itemize}

\cvsection{Volunteer Experience}

\cvevent{Frontend Developer}{\href{https://sirius.utp.edu.co/jointdeveloper/}{jointDeveloper}}{February 2016 -- April 2016}{Pereira, Colombia}
\begin{itemize}

\item Colaborator of the first implementation of the website for jointDeveloper with the use of HTML / CSS and Javascript.

\item Construction of the \textit{Events Section}, incluiding the photo gallery visualization. 
    
\end{itemize}

\divider

\cvevent{Co-founder}{\href{https://sirius.utp.edu.co/jointdeveloper/}{jointDeveloper}}{Since January 2015}{Pereira, Colombia}
\begin{itemize}

\item Comunity and seedbed that seeks to generate interest in the areas of programming and technology in women.

\item Co-founder, mentor and speaker. Programming and technology teaching.

\end{itemize}

\divider

\medskip

% \cvsection{A Day of My Life}

% % Adapted from @Jake's answer from http://tex.stackexchange.com/a/82729/226
% % \wheelchart{outer radius}{inner radius}{
% % comma-separated list of value/text width/color/detail}
% \wheelchart{1.5cm}{0.5cm}{%
%   6/8em/accent!30/{Sleep,\\beautiful sleep}, 
%   4/8em/accent!8/{Watching news},
%   3/7em/accent!8/Meeting &Talking with friends,
%   8/8em/accent!55/Day time Job,
%   2/10em/accent!10/Sports and relaxation,
%   5/6em/accent!20/Spending time with family
% }

% \clearpage
% \cvsection[page2sidebar]{Publications}

% \nocite{*}

% \printbibliography[heading=pubtype,title={\printinfo{\faBook}{Books}},type=book]

% \divider

% \printbibliography[heading=pubtype,title={\printinfo{\faFileTextO}{Journal Articles}},type=article]

% \divider

% \printbibliography[heading=pubtype,title={\printinfo{\faGroup}{Conference Proceedings}},type=inproceedings]

%% If the NEXT page doesn't start with a \cvsection but you'd
%% still like to add a sidebar, then use this command on THIS
%% page to add it. The optional argument lets you pull up the 
%% sidebar a bit so that it looks aligned with the top of the
%% main column.
% \addnextpagesidebar[-1ex]{page3sidebar}

\end{document}
